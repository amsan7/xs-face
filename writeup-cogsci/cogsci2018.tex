% Template for Cogsci submission with R Markdown

% Stuff changed from original Markdown PLOS Template
\documentclass[10pt, letterpaper]{article}

\usepackage{cogsci}
\usepackage{pslatex}
\usepackage{float}
\usepackage{caption}

% amsmath package, useful for mathematical formulas
\usepackage{amsmath}

% amssymb package, useful for mathematical symbols
\usepackage{amssymb}

% hyperref package, useful for hyperlinks
\usepackage{hyperref}

% graphicx package, useful for including eps and pdf graphics
% include graphics with the command \includegraphics
\usepackage{graphicx}

% Sweave(-like)
\usepackage{fancyvrb}
\DefineVerbatimEnvironment{Sinput}{Verbatim}{fontshape=sl}
\DefineVerbatimEnvironment{Soutput}{Verbatim}{}
\DefineVerbatimEnvironment{Scode}{Verbatim}{fontshape=sl}
\newenvironment{Schunk}{}{}
\DefineVerbatimEnvironment{Code}{Verbatim}{}
\DefineVerbatimEnvironment{CodeInput}{Verbatim}{fontshape=sl}
\DefineVerbatimEnvironment{CodeOutput}{Verbatim}{}
\newenvironment{CodeChunk}{}{}

% cite package, to clean up citations in the main text. Do not remove.
\usepackage{cite}

\usepackage{color}

% Use doublespacing - comment out for single spacing
%\usepackage{setspace}
%\doublespacing


% % Text layout
% \topmargin 0.0cm
% \oddsidemargin 0.5cm
% \evensidemargin 0.5cm
% \textwidth 16cm
% \textheight 21cm

\title{Developmental and postural changes in children's visual access to faces}


\author{{\large \bf Alessandro Sacnhez} \\ \texttt{author1@university.edu} \\ Department of Psychology \\ Stanford University \And {\large \bf Bria Long} \\ \texttt{bria@stanford.edu} \\ Department of Psychology \\ Stanford University
    \And {\large \bf Ally Kraus} \\ \texttt{bria@stanford.edu} \\ Department of Psychology \\ Stanford University
    \And {\large \bf Michael C. Frank} \\ \texttt{mcfrank@stanford.edu} \\ Department of Psychology \\ Stanford University}

\begin{document}

\maketitle

\begin{abstract}
The faces of other people are a critical information source for young
children. During early development, children undergo significant
postural and locomotor development, changing from lying and sitting
infants to toddlers who walk independently. We used a head-mounted
camera in conjunction with a face-detection system to explore the
effects of these changes on children's visual access to their
caregivers' faces during an in-lab play session. In a cross-sectional
sample of 8--16 month old children, we found substantial changes in face
accessibility based on age and posture. These changes may translate into
changes in the accessibility of social information during language
learning. We make our corpus available for reanalysis, and discuss
strengths and limitations of the headcam method more generally.

\textbf{Keywords:}
social cognition; face-perception; infancy; locomotion; head-cameras
\end{abstract}

\section{Methods}\label{methods}

\subsection{Participants}\label{participants}

\begin{table}[ht]
\centering
\begin{tabular}{rrrrrr}
  \hline
Age group & N & \% included & Mean age & Video length (min) & \% female \\ 
  \hline
   8 &  12 & 0.46 & 8.71 & 14.41 & 0.50 \\ 
   12 &  12 & 0.40 & 12.62 & 13.48 & 0.58 \\ 
   16 &  12 & 0.31 & 16.29 & 15.00 & 0.50 \\ 
   \hline
\end{tabular}
\caption{\label{tab:pop} Demographics by age group.}
\end{table}

Our final sample consisted of 36 infants and children, distributed into
each of three age groups: 8 months, 12 months, and 16 months.
Participants were recruited from the surrounding community via state
birth records. Participants had no documented disabilities and were
reported to hear at least 80\% English at home. Demographics and
exclusion rates are given in Table \ref{tab:pop}.

In order to compile this final sample, we tested a substantially larger
group of children, who were excluded for the following reasons: 20 for
technical issues related to the headcam, 15 for failing to wear the
headcam, 10 for fewer than 4 minutes of headcam footage, 5 for having
multiple adults present, 5 for missing CDI data, 2 for missing scene
camera footage, 1 for fussiness, and one excluded for sample symmetry.
All inclusion decisions were made independent of the results of
subsequent analyses. ```

\section{Acknowledgements}\label{acknowledgements}

``Thanks to Ally Kraus, Kathy Woo, Aditi Maliwal, and other members of
the Language and Cognition Lab for help in recruitment, data collection,
and annotation. This research was supported by a John Merck Scholars
grant to MCF. An earlier version of this work was presented to the
Cognitive Science Society in Frank, Simmons, Yurovsky, \& Pusiol (2013).
Please address correspondence to Michael C. Frank, Department of
Psychology, Stanford University, 450 Serra Mall (Jordan Hall), Stanford,
CA, 94305, tel: (650) 724-4003, email: \texttt{mcfrank@stanford.edu}.''

\section{References}\label{references}

\setlength{\parindent}{-0.1in} \setlength{\leftskip}{0.125in} \noindent

\hypertarget{refs}{}
\hypertarget{ref-frank2013}{}
Frank, M. C., Simmons, K., Yurovsky, D., \& Pusiol, G. (2013).
Developmental and postural changes in children’s visual access to faces.
In \emph{Proceedings of the 35th annual meeting of the cognitive science
society} (pp. 454--459).

\end{document}
